\chapter*{Resumen}

Actualmente la investigación y desarrollo en robótica móvil está en pleno auge y las cámaras se utilizan cada vez con más frecuencia en robots. Una de las funcionalidades más importantes que se persigue, es que los robots móviles puedan desplazarse por su entorno y navegar desde la posición A a la posición B de forma autónoma. Hoy en día ya existe una familia de técnicas que permite al robot navegar de manera autónoma por zonas desconocidas para él, esta técnica se llama Visual SLAM (\textit{Simultaneous Localization and Mapping}) y aporta al robot la capacidad de autolocalizarse en tiempo real a partir de las imágenes de la cámara en el robot. 

Los algoritmos de Visual SLAM se componen de varias fases, dependiendo de como esté diseñada cada fase el algoritmo proporcionará una posición de modo más o menos preciso y robusto y generará un mapa con mayor o menor exactitud o con mayor o menor velocidad. Sería conveniente disponer de alguna herramienta que permita comparar cuantitativamente la calidad o los errores de precisión de cada algoritmo de tal forma que ayude a encontrar algoritmos óptimos. En este proyecto se ha creado la herramienta SLAMTestbed, que ha sido diseñada para que mida el error de la misma trayectoria calculada por distintos algoritmos.

SLAMTestbed incluye varios módulos de estimación de componentes principales, de offset, de escala, estimadores de rotaciones y traslaciones 3D. Todos ellos permiten llevar al mismo sistema de referencia espacial y al mismo sistema de referencia temporal dos secuencias de posiciones 3D orientadas, de modo que sean comparables y se pueda medir las diferencias entre ellas. Cuando una de estas secuencias es la verdadera y otra la estimada por un algoritmo Visual SLAM entonces esas diferencias son el error del algoritmo.